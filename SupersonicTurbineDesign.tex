\documentclass[a4j,10pt,oneside,openany]{jsbook}
%
\usepackage{amsmath,amssymb}
\usepackage{bm}
\usepackage{graphicx}
\usepackage{ascmac}
\usepackage{makeidx}
%
\makeindex
%
\newcommand{\diff}{\mathrm{d}}  %微分記号
\newcommand{\divergence}{\mathrm{div}\,}  %ダイバージェンス
\newcommand{\grad}{\mathrm{grad}\,}  %グラディエント
\newcommand{\rot}{\mathrm{rot}\,}  %ローテーション
%
\setlength{\textwidth}{\fullwidth}
\setlength{\textheight}{44\baselineskip}
\addtolength{\textheight}{\topskip}
\setlength{\voffset}{-0.6in}
%
\title{{\Huge \textbf{超音速衝動タービンの設計方法}}\\ {\small Ver. 0.1.0}}

\author{Arthur\\ \texttt{Supersonic Turbine Design}}
\date{\today}
\begin{document}
%
%
\maketitle
\frontmatter
\tableofcontents
%
%
\mainmatter

\chapter{設計手法について}
\begin{abstract}

このドキュメントは超音速衝動タービンの翼設計を行えるようにまとめたものである。
主に〜と〜による論文を参考にし、その設計方法に従う。
最後には設計を行えるプログラムとその使用方法について述べる。
はじめに断っておくが、本ドキュメントによる設計手法では決め打ちのパラメータが多々存在し、
最適解になるわけではないため、効率の良いタービンブレードを作成するためには、
実際に何個もブレードを試作し、動かし、効率良のかったタービンブレードのパラメータを
採用する方式を取らざるを得ない。タービンの効率は最大でも8割程度であるため、ロケットの推進剤、燃焼圧ひいては再生冷却溝による圧力ドロップを考慮して設計されたい。
衝撃波打ち消し曲線はインクリメンタル的に計算しなければいけないため、プログラム化するのが望ましい。
このドキュメントでは境界層補正は入れないものとする。

このドキュメントの立ち位置はあくまでも筆者自身が何をやっているかを忘れないように日本語でまとめたものである。
このドキュメントおよびプログラムを使用し、損害を負っても、筆者は責任を負わないこととする。
実験をするにあたっては細心の注意を払い、機械的な知見を持って望むことを期待する。

\end{abstract}

\section{超音速衝動タービンの設計に当たって}
超音速衝動タービンの設計はタービン翼入り口で発生する衝撃波を打ち消すように設計するため、
必要とする変数は限られてくる。そのためタービン出力などのパラメータに左右されず翼設計が行える。

最低限必要であるパラメータは入力マッハ数Mと入力ガス比熱gammaである。
ガス比熱はNASAから提供されているNasaCEAを使用する。OnlineCEAに関して、巻末であるおまけページに記載する。

設計の段階を以下に記述する
1.ブレードの上面、下面の衝撃波打ち消し曲線を作る。
2.円弧区間を作る。
3.ブレード上面の直線区間を作る。
4.ブレード高さを算出し、座標を出力、および描画を行う。
(おまけ)
5.ブレードのエッジを落とし、かつ効率を良くする。
6.パラメータに対しての評価関数。
7.プログラムの使い方。

以上である。適宜必要な数式を引用するので、引用元を参照されたい。
また、インレットとエグジットでマッハ数が変化するため、非対称ブレートについても言及する。

\section{変数}
\dots

\section{衝撃波打ち消し曲線}


各ブレードの極座標関数は以下に成る。


ここでconstantは初期に生成された角度である。
符号の違いは圧縮波と膨張波を表す。
ここでR*を1とした場合の極座標関数を円周上に回転させたものを図〜に示す。
また、比熱比gammaを1.4とする。




衝撃波打ち消し曲線による上面、下面の求め方は初期条件が異なるだけでほぼ同一である。


\section{円弧区間}

衝撃波打ち消し曲線が生成されたのであれば残りの作業は簡単で、つじつまが合うように曲線を繋ぎ合わせるだけである。


\section{直線区間}

ブレード上面の直線区間はインレットのガス入力角度と同じで、衝撃波打ち消し曲線と接続し、ブレード下面のx座標で切り落とすだけである。
イメージ図を以下の図に示す。


\section{ブレード高さ}

直線区間も作成し終わった段階でほとんどのプロセスは終了したが、ブレード上面またはブレード下面を接続しなければならない。
ブレード高さはブレードインレットの上面と下面の差である。
そのため、y軸方向の差を算出し、垂直に移動すればブレードの完成である。

\section{ブレードエッジの切り落とし}

ここまでで作成されたブレードエッジは角度が急すぎて機械的剛性が持たないため、実用性が無い。簡単な方法は負荷に耐えられるところまで
エッジを丸めて運用するだけなのであるが、ブレード下面の打ち消し曲線が消えるため性能が落ち込む。これを解決するために、エッジの切り落としを行う。

筆者にも謎であるが、決め打ちのパラメータがあるためそれに従う。

ブレード直線区間に対して切り落とし直線は8度ほどの角度をつけて、高さ方向にはブレード高さに対して1割ほどの高さに設定する。

ここで考慮しなければいけない項目は、超音速流を角度の変わる壁面に当てると、衝撃波を生じ、速度が変化することである。
ブレードエッジの切り落としでどの程度ガスが加速されるかを考慮して最初からブレードの設計のし直しを行わないといけない。
が、考え方が変わるわけではないので、プログラムの力を使ってごり押しで計算を行う。

\section{ブレード曲線とパラメータに対する評価関数と制限}

\section{プログラムの使いかた}

基本的には入力ガスのマッハ数と比熱比だけでいいのであるが、
様々な制限があるため、入力側に角度変化の下限、上限も記入する必要がある。
また、アシンメトリカルなブレードジオメトリにする場合はエグジットのパラメータも入力する必要がある。

\chapter{衝撃波打ち消し曲線}
\begin{abstract}
超音速タービンの翼設計で一番最初に行うのが衝撃波を打ち消す曲線を作ることである。
この作業が終わると別の翼形状は自動的に決まってくる。
\end{abstract}

\section{変数リスト}
基本的に超音速衝動タービンを設計するに当たって必要な物理パラメータは、
ガス比熱とガス入射角度のみである\\

ではリストアップする。

\begin{thebibliography}{20}
\bibitem{...}...
...
\end{thebibliography}

\newpage
\printindex
%
%
\end{document}
